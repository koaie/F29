

% This is a simple sample document.  For more complicated documents take a look in the exercise tab. Note that everything that comes after a % symbol is treated as comment and ignored when the code is compiled.

\documentclass{article} % \documentclass{} is the first command in any LaTeX code.  It is used to define what kind of document you are creating such as an article or a book, and begins the document preamble

\usepackage{amsmath} % \usepackage is a command that allows you to add functionality to your LaTeX code
\usepackage{tikz}
\usepackage{enumitem}
\usepackage{varwidth}
\usepackage{tasks}
\usepackage[T1]{fontenc}

\usetikzlibrary{automata, positioning}

\newlength{\drop}
% The preamble ends with the command \begin{document}
\begin{document} % All begin commands must be paired with an end command somewhere
    \begin{titlepage}
        \drop=0.1\textheight
        \centering
        \vspace*{\baselineskip}
        {\LARGE Mod/Div Turing Machine }\\[0.2\baselineskip]
        \rule{\textwidth}{0.4pt}\vspace*{-\baselineskip}\vspace{3.2pt}
        \rule{\textwidth}{1.6pt}\\[\baselineskip]
        \scshape
        Course Work\\
        F29FB\par
        \vfill
        Submitted by \\[\baselineskip]
        {\Large Yoav Levi\par}
        {\itshape H00347035\par}
        \vspace*{8\baselineskip}
    \end{titlepage}

    \section{/Graph/} % creates a section
        \begin{tikzpicture} [draw=cyan!70!black,
            node distance = 4cm, 
            on grid, 
            auto]
        
        % State S
        \node (s) [state, 
            initial,  
            initial text = {}] {$s$};

        % State q0 
        \node (q0) [state, 
            right = of s] {$q_0$};
        
        % State q1    
        \node (q1) [state,
            accepting,
            above right = of q0] {$q_1$};

        % State q2   
        \node (q2) [state, 
         accepting,
         below right = of q0] {$q_2$};
        
        % State q3    
        \node (q3) [state,
            below right = of q1] {$q_3$};
        
        
        
        % Arrows
        \path [-stealth, thick]
            (s) edge [bend left] node {$\epsilon,\epsilon,Z_0$} (q0)
            (q0) edge [bend left] node {$b,\epsilon,b$} (q1)
            (q0) edge [bend right] node {$a,\epsilon,a$} (q2)
            (q1) edge [bend left] node {$\epsilon,Z_0,\epsilon$} (q3)
            (q2) edge [bend right ] node {$\epsilon,Z_0,\epsilon$} (q3)
        
            (q1) edge [loop above]  node {$a,b,\epsilon$}()
            (q1) edge [loop below]  node {$b,\epsilon,b$}()

            (q2) edge [loop above]  node {$b,a,\epsilon$}()
            (q2) edge [loop below]  node {$a,\epsilon,a$}()
            ;
        \end{tikzpicture}
\end{document} % This is the end of the document
